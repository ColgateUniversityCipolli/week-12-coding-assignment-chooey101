\documentclass{article}\usepackage[]{graphicx}\usepackage[]{xcolor}
% maxwidth is the original width if it is less than linewidth
% otherwise use linewidth (to make sure the graphics do not exceed the margin)
\makeatletter
\def\maxwidth{ %
  \ifdim\Gin@nat@width>\linewidth
    \linewidth
  \else
    \Gin@nat@width
  \fi
}
\makeatother

\definecolor{fgcolor}{rgb}{0.345, 0.345, 0.345}
\newcommand{\hlnum}[1]{\textcolor[rgb]{0.686,0.059,0.569}{#1}}%
\newcommand{\hlsng}[1]{\textcolor[rgb]{0.192,0.494,0.8}{#1}}%
\newcommand{\hlcom}[1]{\textcolor[rgb]{0.678,0.584,0.686}{\textit{#1}}}%
\newcommand{\hlopt}[1]{\textcolor[rgb]{0,0,0}{#1}}%
\newcommand{\hldef}[1]{\textcolor[rgb]{0.345,0.345,0.345}{#1}}%
\newcommand{\hlkwa}[1]{\textcolor[rgb]{0.161,0.373,0.58}{\textbf{#1}}}%
\newcommand{\hlkwb}[1]{\textcolor[rgb]{0.69,0.353,0.396}{#1}}%
\newcommand{\hlkwc}[1]{\textcolor[rgb]{0.333,0.667,0.333}{#1}}%
\newcommand{\hlkwd}[1]{\textcolor[rgb]{0.737,0.353,0.396}{\textbf{#1}}}%
\let\hlipl\hlkwb

\usepackage{framed}
\makeatletter
\newenvironment{kframe}{%
 \def\at@end@of@kframe{}%
 \ifinner\ifhmode%
  \def\at@end@of@kframe{\end{minipage}}%
  \begin{minipage}{\columnwidth}%
 \fi\fi%
 \def\FrameCommand##1{\hskip\@totalleftmargin \hskip-\fboxsep
 \colorbox{shadecolor}{##1}\hskip-\fboxsep
     % There is no \\@totalrightmargin, so:
     \hskip-\linewidth \hskip-\@totalleftmargin \hskip\columnwidth}%
 \MakeFramed {\advance\hsize-\width
   \@totalleftmargin\z@ \linewidth\hsize
   \@setminipage}}%
 {\par\unskip\endMakeFramed%
 \at@end@of@kframe}
\makeatother

\definecolor{shadecolor}{rgb}{.97, .97, .97}
\definecolor{messagecolor}{rgb}{0, 0, 0}
\definecolor{warningcolor}{rgb}{1, 0, 1}
\definecolor{errorcolor}{rgb}{1, 0, 0}
\newenvironment{knitrout}{}{} % an empty environment to be redefined in TeX

\usepackage{alltt}
\usepackage[margin=1.0in]{geometry} % To set margins
\usepackage{amsmath}  % This allows me to use the align functionality.
                      % If you find yourself trying to replicate
                      % something you found online, ensure you're
                      % loading the necessary packages!
\usepackage{amsfonts} % Math font
\usepackage{fancyvrb}
\usepackage{hyperref} % For including hyperlinks
\usepackage[shortlabels]{enumitem}% For enumerated lists with labels specified
                                  % We had to run tlmgr_install("enumitem") in R
\usepackage{float}    % For telling R where to put a table/figure
\usepackage{natbib}        %For the bibliography
\bibliographystyle{apalike}%For the bibliography
\IfFileExists{upquote.sty}{\usepackage{upquote}}{}
\begin{document}


\begin{enumerate}
%%%%%%%%%%%%%%%%%%%%%%%%%%%%%%%%%%%%%%%%%%%%%%%%%%%%%%%%%%%%%%%%%%%%%%%%%%%%%%%%
%%%%%%%%%%%%%%%%%%%%%%%%%%%%%%%%%%%%%%%%%%%%%%%%%%%%%%%%%%%%%%%%%%%%%%%%%%%%%%%%
% Question 1
%%%%%%%%%%%%%%%%%%%%%%%%%%%%%%%%%%%%%%%%%%%%%%%%%%%%%%%%%%%%%%%%%%%%%%%%%%%%%%%%
%%%%%%%%%%%%%%%%%%%%%%%%%%%%%%%%%%%%%%%%%%%%%%%%%%%%%%%%%%%%%%%%%%%%%%%%%%%%%%%%
\item A group of researchers is running an experiment over the course of 30 months, 
with a single observation collected at the end of each month. Let $X_1, ..., X_{30}$
denote the observations for each month. From prior studies, the researchers know that
\[X_i \sim f_X(x),\]
but the mean $\mu_X$ is unknown, and they wish to conduct the following test
\begin{align*}
H_0&: \mu_X = 0\\
H_a&: \mu_X > 0.
\end{align*}
At month $k$, they have accumulated data $X_1, ..., X_k$ and they have the 
$t$-statistic
\[T_k = \frac{\bar{X} - 0}{S_k/\sqrt{n}}.\]
The initial plan was to test the hypotheses after all data was collected (at the 
end of month 30), at level $\alpha=0.05$. However, conducting the experiment is 
expensive, so the researchers want to ``peek" at the data at the end of month 20 
to see if they can stop it early. That is, the researchers propose to check 
whether $t_{20}$ provides statistically discernible support for the alternative. 
If it does, they will stop the experiment early and report support for the 
researcher's alternative hypothesis. If it does not, they will continue to month 
30 and test whether $t_{30}$ provides statistically discernible support for the
alternative.

\begin{enumerate}
  \item What values of $t_{20}$ provide statistically discernible support for the
  alternative hypothesis? - \textbf{At month 20, because we obtained a critical t-value using the right-tailed test, we can reject the null hypothesis if T20 $>$ $1.729$.}
\begin{knitrout}\scriptsize
\definecolor{shadecolor}{rgb}{0.969, 0.969, 0.969}\color{fgcolor}\begin{kframe}
\begin{alltt}
\hldef{alpha} \hlkwb{=} \hlnum{0.05}
\hldef{t20.crit} \hlkwb{<-} \hlkwd{qt}\hldef{(}\hlnum{1}\hlopt{-}\hldef{alpha,} \hlkwc{df} \hldef{=} \hlnum{20} \hlopt{-} \hlnum{1}\hldef{)}
\hldef{t20.crit}
\end{alltt}
\begin{verbatim}
## [1] 1.729133
\end{verbatim}
\end{kframe}
\end{knitrout}
  \item What values of $t_{30}$ provide statistically discernible support for the
  alternative hypothesis? - \textbf{At month 30, using the same right tailed test to obtain a critical t-value, we can reject the null hypothesis if T30 $>$ $1.699$.}
\begin{knitrout}\scriptsize
\definecolor{shadecolor}{rgb}{0.969, 0.969, 0.969}\color{fgcolor}\begin{kframe}
\begin{alltt}
\hldef{alpha} \hlkwb{=} \hlnum{0.05}
\hldef{t30.crit} \hlkwb{<-} \hlkwd{qt}\hldef{(}\hlnum{1}\hlopt{-}\hldef{alpha,} \hlkwc{df} \hldef{=} \hlnum{30} \hlopt{-} \hlnum{1}\hldef{)}
\hldef{t30.crit}
\end{alltt}
\begin{verbatim}
## [1] 1.699127
\end{verbatim}
\end{kframe}
\end{knitrout}
  \item Suppose $f_X(x)$ is a Laplace distribution with $a=0$ and $b=4.0$.
  Conduct a simulation study to assess the Type I error rate of this approach.\\
  \textbf{Note:} You can use the \texttt{rlaplace()} function from the \texttt{VGAM}
  package for \texttt{R} \citep{VGAM}. - \textbf{By conducting a simulation study under the assumtion $f_X(x)$ is a Laplace distribution with $a=0$ and $b=4.0$, we were able to calculate an overall rate of type 1 errors of $7.19$ Percent. }
  
\begin{knitrout}\scriptsize
\definecolor{shadecolor}{rgb}{0.969, 0.969, 0.969}\color{fgcolor}\begin{kframe}
\begin{alltt}
\hldef{n.sim} \hlkwb{=} \hlnum{10000}
\hlcom{#1c}

\hldef{sim.tbl} \hlkwb{<-} \hlkwd{tibble}\hldef{(}
\hlkwc{early_reject} \hldef{=} \hlkwd{logical}\hldef{(n.sim),}
\hlkwc{overall}      \hldef{=} \hlkwd{logical}\hldef{(n.sim)}
\hldef{)}


\hlkwa{for} \hldef{(i} \hlkwa{in} \hlnum{1}\hlopt{:}\hldef{n.sim) \{}
\hldef{x} \hlkwb{<-} \hlkwd{rlaplace}\hldef{(}\hlnum{30}\hldef{,} \hlkwc{location} \hldef{=} \hlnum{0}\hldef{,} \hlkwc{scale} \hldef{=} \hlnum{4}\hldef{)}

\hlcom{# interim look}
\hldef{t20} \hlkwb{<-} \hlkwd{mean}\hldef{(x[}\hlnum{1}\hlopt{:}\hlnum{20}\hldef{])} \hlopt{/} \hldef{(}\hlkwd{sd}\hldef{(x[}\hlnum{1}\hlopt{:}\hlnum{20}\hldef{])} \hlopt{/} \hlkwd{sqrt}\hldef{(}\hlnum{20}\hldef{))}
\hldef{sim.tbl}\hlopt{$}\hldef{early_reject[i]} \hlkwb{<-} \hldef{(t20} \hlopt{>} \hldef{t20.crit)}

\hlkwa{if} \hldef{(sim.tbl}\hlopt{$}\hldef{early_reject[i]) \{}
  \hlcom{# if rejected early, overall is also TRUE}
  \hldef{sim.tbl}\hlopt{$}\hldef{overall[i]} \hlkwb{<-} \hlnum{TRUE}
\hldef{\}} \hlkwa{else} \hldef{\{}
  \hlcom{# final look}
  \hldef{t30} \hlkwb{<-} \hlkwd{mean}\hldef{(x)} \hlopt{/} \hldef{(}\hlkwd{sd}\hldef{(x)} \hlopt{/} \hlkwd{sqrt}\hldef{(}\hlnum{30}\hldef{))}
  \hldef{sim.tbl}\hlopt{$}\hldef{overall[i]} \hlkwb{<-} \hldef{(t30} \hlopt{>} \hldef{t30.crit)}
\hldef{\}}
\hldef{\}}


\hldef{sim.tbl.rates} \hlkwb{<-} \hldef{sim.tbl |>}
\hlkwd{mutate}\hldef{(}\hlkwc{late_reject} \hldef{=} \hlopt{!}\hldef{early_reject} \hlopt{&} \hldef{overall)|>}
\hlkwd{summarize}\hldef{(}
  \hlkwc{early_rate}   \hldef{=} \hlkwd{mean}\hldef{(early_reject),}
  \hlkwc{late_rate}    \hldef{=} \hlkwd{mean}\hldef{(late_reject),}
  \hlkwc{overall_rate} \hldef{=} \hlkwd{mean}\hldef{(overall)}
\hldef{)}
\hldef{sim.tbl.rates}
\end{alltt}
\begin{verbatim}
## # A tibble: 1 x 3
##   early_rate late_rate overall_rate
##        <dbl>     <dbl>        <dbl>
## 1     0.0491    0.0273       0.0764
\end{verbatim}
\end{kframe}
\end{knitrout}
  \item \textbf{Optional Challenge:} Can you find a value of $\alpha<0.05$ that yields a 
  Type I error rate of 0.05? 

\end{enumerate}
%%%%%%%%%%%%%%%%%%%%%%%%%%%%%%%%%%%%%%%%%%%%%%%%%%%%%%%%%%%%%%%%%%%%%%%%%%%%%%%%
%%%%%%%%%%%%%%%%%%%%%%%%%%%%%%%%%%%%%%%%%%%%%%%%%%%%%%%%%%%%%%%%%%%%%%%%%%%%%%%%
% Question 2
%%%%%%%%%%%%%%%%%%%%%%%%%%%%%%%%%%%%%%%%%%%%%%%%%%%%%%%%%%%%%%%%%%%%%%%%%%%%%%%%
%%%%%%%%%%%%%%%%%%%%%%%%%%%%%%%%%%%%%%%%%%%%%%%%%%%%%%%%%%%%%%%%%%%%%%%%%%%%%%%%
  \item Perform a simulation study to assess the robustness of the $T$ test. 
  Specifically, generate samples of size $n=15$ from the Beta(10,2), Beta(2,10), 
  and Beta(10,10) distributions and conduct the following hypothesis tests against 
  the actual mean for each case (e.g., $\frac{10}{10+2}$, $\frac{2}{10+2}$, and 
  $\frac{10}{10+10}$). 
  

  \begin{enumerate}
    \item What proportion of the time do we make an error of Type I for a
    left-tailed test? - \textbf{The simulation test we have performed below suggests that we make a type I error while using a left-tailed test $3.02$\% of the time for the Beta(10,2) distribution, $7.81$\% of the time for the Beta(2,10) distribution, and $5.18$\% of the time for the Beta(10,10) distribution.}
\begin{knitrout}\scriptsize
\definecolor{shadecolor}{rgb}{0.969, 0.969, 0.969}\color{fgcolor}\begin{kframe}
\begin{alltt}
\hldef{n_sim}      \hlkwb{<-} \hlnum{10000}
\hldef{n}          \hlkwb{<-} \hlnum{15}
\hldef{df}         \hlkwb{<-} \hldef{n} \hlopt{-} \hlnum{1}

\hldef{tcrit_left} \hlkwb{<-} \hlkwd{qt}\hldef{(alpha,} \hlkwc{df}\hldef{=df)}

\hldef{tcrit_right} \hlkwb{<-} \hlkwd{qt}\hldef{(}\hlnum{1} \hlopt{-} \hldef{alpha,} \hlkwc{df} \hldef{= df)}
\hlcom{# two‐sided critical value}
\hldef{tcrit_2s}   \hlkwb{<-} \hlkwd{qt}\hldef{(}\hlnum{1} \hlopt{-} \hldef{alpha}\hlopt{/}\hlnum{2}\hldef{,} \hlkwc{df} \hldef{= df)}

\hlcom{#List of parameters for each distribution}
\hldef{params} \hlkwb{<-} \hlkwd{list}\hldef{(}
\hlkwc{Beta_10_2}  \hldef{=} \hlkwd{list}\hldef{(}\hlkwc{a}\hldef{=}\hlnum{10}\hldef{,} \hlkwc{b}\hldef{=}\hlnum{2}\hldef{,}  \hlkwc{mu}\hldef{=}\hlnum{10}\hlopt{/}\hlnum{12}\hldef{),}
\hlkwc{Beta_2_10}  \hldef{=} \hlkwd{list}\hldef{(}\hlkwc{a}\hldef{=}\hlnum{2}\hldef{,}  \hlkwc{b}\hldef{=}\hlnum{10}\hldef{,} \hlkwc{mu}\hldef{=}\hlnum{2}\hlopt{/}\hlnum{12}\hldef{),}
\hlkwc{Beta_10_10} \hldef{=} \hlkwd{list}\hldef{(}\hlkwc{a}\hldef{=}\hlnum{10}\hldef{,} \hlkwc{b}\hldef{=}\hlnum{10}\hldef{,}\hlkwc{mu}\hldef{=}\hlnum{0.5}\hldef{)}
\hldef{)}

\hlcom{####Rejection rate when using a left tailed test}
\hldef{type1_rates_left} \hlkwb{<-} \hlkwd{sapply}\hldef{(params,} \hlkwa{function}\hldef{(}\hlkwc{p}\hldef{) \{}
\hldef{rejections} \hlkwb{<-} \hlkwd{replicate}\hldef{(n_sim, \{}
  \hldef{x}     \hlkwb{<-} \hlkwd{rbeta}\hldef{(n, p}\hlopt{$}\hldef{a, p}\hlopt{$}\hldef{b)}
  \hldef{tstat} \hlkwb{<-} \hldef{(}\hlkwd{mean}\hldef{(x)} \hlopt{-} \hldef{p}\hlopt{$}\hldef{mu)} \hlopt{/} \hldef{(}\hlkwd{sd}\hldef{(x)} \hlopt{/} \hlkwd{sqrt}\hldef{(n))}
  \hldef{tstat} \hlopt{<} \hldef{tcrit_left}
\hldef{\})}
\hlkwd{mean}\hldef{(rejections)}
\hldef{\})}
\hldef{type1_rates_left}
\end{alltt}
\begin{verbatim}
##  Beta_10_2  Beta_2_10 Beta_10_10 
##     0.0291     0.0766     0.0498
\end{verbatim}
\end{kframe}
\end{knitrout}
    \item What proportion of the time do we make an error of Type I for a
    right-tailed test? - \textbf{The simulation test we have performed below suggests that we make a type I error while using a right-tailed test $7.74$\% of the time for the Beta(10,2) distribution, $2.75$\% of the time for the Beta(2,10) distribution, and $4.85$\% of the time for the Beta(10,10) distribution.}
\begin{knitrout}\scriptsize
\definecolor{shadecolor}{rgb}{0.969, 0.969, 0.969}\color{fgcolor}\begin{kframe}
\begin{alltt}
\hldef{type1_rates_right} \hlkwb{<-} \hlkwd{sapply}\hldef{(params,} \hlkwa{function}\hldef{(}\hlkwc{p}\hldef{) \{}
\hldef{rejections} \hlkwb{<-} \hlkwd{replicate}\hldef{(n_sim, \{}
  \hldef{x}     \hlkwb{<-} \hlkwd{rbeta}\hldef{(n, p}\hlopt{$}\hldef{a, p}\hlopt{$}\hldef{b)}
  \hldef{tstat} \hlkwb{<-} \hldef{(}\hlkwd{mean}\hldef{(x)} \hlopt{-} \hldef{p}\hlopt{$}\hldef{mu)} \hlopt{/} \hldef{(}\hlkwd{sd}\hldef{(x)} \hlopt{/} \hlkwd{sqrt}\hldef{(n))}
  \hlcom{# reject if t-statistic is too large}
  \hldef{tstat} \hlopt{>} \hldef{tcrit_right}
\hldef{\})}
\hlkwd{mean}\hldef{(rejections)}
\hldef{\})}
\hldef{type1_rates_right}
\end{alltt}
\begin{verbatim}
##  Beta_10_2  Beta_2_10 Beta_10_10 
##     0.0776     0.0273     0.0469
\end{verbatim}
\end{kframe}
\end{knitrout}
    \item What proportion of the time do we make an error of Type I for a
    two-tailed test? - \textbf{The simulation test we have performed below suggests that we make a type I error while using a two-tailed test $5.75$\% of the time for the Beta(10,2) distribution, $6.28$\% of the time for the Beta(2,10) distribution, and $5.08$\% of the time for the Beta(10,10) distribution.}
\begin{knitrout}\scriptsize
\definecolor{shadecolor}{rgb}{0.969, 0.969, 0.969}\color{fgcolor}\begin{kframe}
\begin{alltt}
\hldef{type1_rates_two} \hlkwb{<-} \hlkwd{sapply}\hldef{(params,} \hlkwa{function}\hldef{(}\hlkwc{p}\hldef{) \{}
  \hlkwd{mean}\hldef{(}\hlkwd{replicate}\hldef{(n_sim, \{}
    \hldef{x}     \hlkwb{<-} \hlkwd{rbeta}\hldef{(n, p}\hlopt{$}\hldef{a, p}\hlopt{$}\hldef{b)}
    \hldef{tstat} \hlkwb{<-} \hldef{(}\hlkwd{mean}\hldef{(x)} \hlopt{-} \hldef{p}\hlopt{$}\hldef{mu)} \hlopt{/} \hldef{(}\hlkwd{sd}\hldef{(x)} \hlopt{/} \hlkwd{sqrt}\hldef{(n))}
    \hldef{(tstat} \hlopt{< -}\hldef{tcrit_2s)} \hlopt{|} \hldef{(tstat} \hlopt{> +}\hldef{tcrit_2s)}
  \hldef{\}))}
\hldef{\})}
\hldef{type1_rates_two}
\end{alltt}
\begin{verbatim}
##  Beta_10_2  Beta_2_10 Beta_10_10 
##     0.0603     0.0602     0.0551
\end{verbatim}
\end{kframe}
\end{knitrout}
    \item How does skewness of the underlying population distribution effect
    Type I error across the test types? - \textbf{The direction skewness of each distribution appears to have a positive coorelation with the direction of the tail of each test, as the rate of type I errors occurring is the lowest for each distribution in the same direction that each distribution is skewed.}
    \begin{table}[ht]
\centering
\begin{tabular}{lrrr}
  \hline
Distribution & Left\_Tailed & Right\_Tailed & Two\_Tailed \\ 
  \hline
Beta(10,2) & 0.0302 & 0.0781 & 0.0518 \\ 
  Beta(2,10) & 0.0774 & 0.0275 & 0.0485 \\ 
  Beta(10,10) & 0.0575 & 0.0628 & 0.0508 \\ 
   \hline
\end{tabular}
\caption{Empirical Type I Error Rates by Test Type and Distribution} 
\label{tab:type1}
\end{table}
  \end{enumerate}
%%%%%%%%%%%%%%%%%%%%%%%%%%%%%%%%%%%%%%%%%%%%%%%%%%%%%%%%%%%%%%%%%%%%%%%%%%%%%%%%
%%%%%%%%%%%%%%%%%%%%%%%%%%%%%%%%%%%%%%%%%%%%%%%%%%%%%%%%%%%%%%%%%%%%%%%%%%%%%%%%
% End Document
%%%%%%%%%%%%%%%%%%%%%%%%%%%%%%%%%%%%%%%%%%%%%%%%%%%%%%%%%%%%%%%%%%%%%%%%%%%%%%%%
%%%%%%%%%%%%%%%%%%%%%%%%%%%%%%%%%%%%%%%%%%%%%%%%%%%%%%%%%%%%%%%%%%%%%%%%%%%%%%%%
\end{enumerate}
\bibliography{bibliography}
\end{document}
